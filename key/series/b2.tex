\documentclass{article}
\usepackage{amsmath}

\begin{document}

\noindent
Does $\displaystyle \sum_{n=1}^\infty \frac{(-1)^n}n$
diverge, converge absolutely, or converge conditionally?

\begin{itemize}

\item {\bf Solution.} The series $\displaystyle \sum_{n=1}^\infty \frac{(-1)^n}n$ is alternating.

Let $a_n = \frac{(-1)^n}{n}$. Then $b_n = |a_n| = \frac1n$. The sequence $b_n$ is decreasing and 
\[ \lim_{n \rightarrow \infty} b_n = 0.\]

By the Alternating Series Test,
$\displaystyle \sum_{n=1}^\infty \frac{(-1)^n}n$
converges.

Does $\displaystyle \sum_{n=1}^\infty \frac{(-1)^n}n$ converge absolutely or conditionally?
We study $\sum |a_n|$, namely $\displaystyle \sum_{n=1}^\infty \left|\frac{(-1)^n}n\right|$, which is the series 
$\displaystyle \sum_{n=1}^\infty \frac{1}n$, which diverges by the $p$-series test, so 
$\displaystyle \sum_{n=1}^\infty \frac{(-1)^n}n$ converges conditionally.

\end{itemize}

\end{document}
