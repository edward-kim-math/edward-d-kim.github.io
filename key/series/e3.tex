\documentclass{article}
\usepackage{amsmath}
\usepackage[margin=1.0in]{geometry}
\usepackage{xcolor}

\begin{document}

\noindent
Does $\displaystyle \sum_{n=2}^\infty \frac{1}{n (\ln n)^{100}}$
diverge, converge absolutely, or converge conditionally?


\subsection*{Solution}

The function $f(x)=\frac{1}{x (\ln x)^{100}}$ is continuous, positive, and decreasing on $[3,\infty)$.

We do the following indefinite integral using the substitution $u= \ln x$, so $du = \frac1x\,dx$:
\[
\int \frac{1}{x (\ln x)^{100}}\,dx
= \int u^{-100}\,du = \frac{-1}{99u^{99}} + C = \frac{-1}{99(\ln x)^{99}} + C
\]
so
\begin{align*}
\int_3^\infty \frac1{x (\ln x)^{100}}\,dx 
&= \lim_{t \to \infty} \int_3^t \frac1{x(\ln x)^{100}}\,dx\\
&= \lim_{t \to \infty}\left[   \frac{-1}{99(\ln t)^{99}} -  \frac{-1}{99(\ln 3)^{99}}  \right] \\
&= \frac{1}{99(\ln 3)^{99}}
\end{align*}
Since the integral $\displaystyle  \int_3^\infty \frac{1}{x \ln x}\,dx$ converges, the series  $\displaystyle \sum_{n=3}^\infty \frac{1}{n (\ln n)^{100}}$ converges by the Integral Test. So the series $\displaystyle \sum_{n=2}^\infty \frac{1}{n \ln n}$ converges as well.

Since all terms (except when $n=2$) are positive, we essentially have  $\sum |a_n| = \sum a_n$, so the series $\displaystyle \sum_{n=2}^\infty \frac{1}{n (\ln n)^{100}}$ converges absolutely.


\end{document}%%%%%%%%%%%%%%%%%

\begin{align*}
L&=\lim_{n \to \infty} \sqrt[n]{|a_n|}\\
&= \lim_{n \to \infty} \sqrt[n]{\left| \right|}\\
\end{align*}


Since $\sum |a_n| = \sum a_n$, the series $\displaystyle \sum_{n=1}^\infty AAAAAAAAAAAAAA$ converges absolutely.

Since $|r| < 1$, the series ...  converges by the Geometric Series Test.

Since $|r| \geq 1$, the series ...  diverges by the Geometric Series Test.

The function $f(x)=\frac{}{}$ is continuous, positive, and decreasing on $[1,\infty)$.

\subsection*{Solution}

